%!TEX root =./tarea2.tex

\begin{enumerate}[a)]
    \item Para demostrar que la relación es de equivalencia demostraremos que es refleja, simetrica y transitiva.
    
    \begin{enumerate}[1)]
        \item $X \sim_n X \iff n + 1$ $|$ $x + nx$ lo cual es trivial ya que podemos escribir $x + nx$ como $x (n+1))$, y $n + 1$ siempre divide a $x (n+1)$

        \item Para demostrar simetría debemos demostrar que $X \sim_n Y \implies Y \sim_n X$
        \begin{align*}
            (n+1)k &= x + ny \\
                   &= x + (n+1)y - y\\
            (n+1)(k-y) &= x - y\\
        \end{align*}
        ahora si nos fijamos en $Y \sim_n X$, tenemos que
        
        \begin{align*}
            y + nx &= y + (n+1)x - x \\
                   &= (n+1)x + (n+1)(y-k)\\
                   &= (n+1)(x + y - k)\\
        \end{align*}
        
        por lo que $(n+1)$ $|$ $x+ny$ $\implies$ $(n+1)$ $|$ $y + nx$ y la relación es simetrica.
        
        \item La relación es transitiva si $X \sim_n Y \land Y \sim_n Z \implies X \sim_n Z$.
        
        Para demostrar que la relación es transitiva haremos algo similar.
        
        Sabemos que $n+1$ $|$ $x + ny$ y que $n+1$ $|$ $y + nz$. Factorizando de igual manera que en la demostración anterior:
        
        \hspace{10mm} $(n+1)(k - y) = x - y$ y $(n+1)(c - z) = y - z$
        
        Si tenemos la ecuación $x + nz$, podemos escribirla como $x + (n+1)z - z$, y reemplazando por las ecuaciones anteriores, nos queda finalmente 
        $(n+1)(k - y) + y + (n+1)z - y + (n+1)(c - z)$, factorizando por $(n + 1)$, nos queda $(n+1)(k - y + c)$, entonces $x + nz$ es multiplo de $(n+1)$, por lo que se cumple la implicancia y la relación es de equivalencia.
    \end{enumerate}
    
    
    
    \item Para probar equivalencia se debe demostrar que la relación es refleja, simetrica y transitiva.
    
    \begin{enumerate}[1)]
        \item Para demostrar que es refleja, tenemos que 
        \begin{align*}
            X \sim X \iff (X \cup X - X \cap X) \subseteq S
        \end{align*}
        
        lo cual solo lo satisface el conjunto vacío. Por definición,
        \begin{align*}
            \emptyset \subseteq S
        \end{align*}
        
        \item Para demostrar que es simetrica, tenemos que 
        \begin{align*}
            X \sim Y \iff (X \cup Y - X \cap Y) \subseteq S
        \end{align*}
        
        pero por asociatividad de los operadores $\cap$ y $\cup$, también tenemos que
        \begin{align*}
            X \sim Y \iff (Y \cup X - Y \cap X) \subseteq S
        \end{align*}
        o sea,
        \begin{align*}
            Y \sim X \iff (Y \cup X - Y \cap X) \subseteq S
        \end{align*}
    
        \item Para demostrar transitividad, se debe probar que para $X$, $Y$, $Z$ cualquieras,
        
        \begin{align*}
            X \sim Y \land Y \sim Z \implies X \sim Z
        \end{align*}
        se cumple.
        
        en el lado izquierdo tenemos que para un $x$ cualquiera en  $\mathcal{U}$, se que cumple que
        
        $(x \in X \land x \notin Y \lor x \notin X \land x \in Y) \lor (x \in Y \land x \notin Z \lor x \notin Y \land x \in Z)$
        
        para que $X \sim Y \land X \sim Z$ sea cierto (excluimos el caso $x \in X \land x \in Z$ pues no nos sirve).
        
        tenemos que demostrar que lo de la derecha es verdadero.
        
        Para el mismo $x$, si $x \in X \land x \notin Z$ podemos ver que si $x \in Y$, se cumple pues $x \in Y \land x \notin Z$, por lo que $1 \implies 1$, y si $x \in Y$ también se cumple pues $x \in Y \land x \notin Z$, por lo que de nuevo $1 \implies 1$.
        
        Para el caso $x \notin X \land x \in Z$, podemos ver que si $x \in Y$ se cumple, ya que $x \notin X \land x \in Y$, y similarmente si $x \notin Y$, entonces $x \notin Y \land x \in Z$.

        Para todos los casos donde se cumple el lado izquierdo, el lado derecho también se cumple. Por lo que podemos decir que el lado izquierdo implica el derecho. 
        
        El único caso donde no se cumple es para algun $x$ que pertenezca a $X$ y a $Z$, pero ese caso no nos sirve pues por definición del operador $\sim$ no puede ocurrir.
        
        \end{enumerate}
        
\end{enumerate}

