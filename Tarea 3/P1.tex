%!TEX root =./tarea2.tex

\newcommand\myeq{\stackrel{\mathclap{\scriptsize\mbox{HI}}}{\equiv}}

\begin{enumerate}[a)]
    
    \item Ocupando la definición de suma vista en clases:
    
    \hspace{10mm} 1. $sum(m, 0) = m$
    
    \hspace{10mm} 2. $sum(m, \sigma(n)) =\sigma(sum(m, n))$
    
    Es trivial concluir que el 0 es elemento neutro.
    
    \vspace{5mm}
    
    \item Ocupando la definición de multiplicación:
    
    \hspace{10mm} 1. $mult(m, 0) = 0$
    
    \hspace{10mm} 2. $mult(m, \sigma(n)) =sum(m, mult(m, n))$
    
    notemos que 
    \begin{align*}
        \hspace{-20mm} mult(m, 1) &= mult(m, \sigma(0)) \\
                   &= sum(m, mult(m, 0)) \\
                   &= sum(m, 0) = m
    \end{align*}
    
    Por lo que $1$ es elemento neutro de la multiplicación
    
    
    \item Queremos demostrar que 

    \hspace{20mm} $mult(a, sum(m, n)) = sum(mult(a, m), mult(a, n))$
    
    para cualquier $a, m, n$.
    
    Por inducción estructural:
    
    \textbf{BI}: Para $n = 0$ tenemos que
    \begin{align*}
        mult(a, sum(m, 0)) &= mult(a, m) \\
                           &= sum(mult(a, m), 0))\\
                           &= sum(mult(a, m), mult(a, 0))
    \end{align*}
    
    por lo que el caso base se cumple.
    
    \newpage
    
    \textbf{HI}: Supongamos que se cumple que 
    \begin{align*}
        mult(a, sum(m, n)) = sum(mult(a, m), mult(a, n))
    \end{align*}
    
    \textbf{TI}: demostrar que tambien se cumple que
    \begin{align*}
        mult(a, sum(m, \sigma(n))) &= sum(mult(a, m), mult(a, \sigma(n)) \\
                                   &= sum(mult(a, m), sum(a, mult(a, n)))
    \end{align*}
    
    Partiremos por el lado izquierdo:
    
    \begin{align*}
        mult(a, sum(m, \sigma(n))) &= mult(a, \sigma(sum(m, n))) \\
                                   &= sum(a, mult(a, sum(m, n))) \\
                                   &\myeq sum(a, sum(mult(a, m), mult(a, n))) \\
                                   &= sum(mult(a, m), sum(a, mult(a, n))) \text{ por asociatividad} \\
    \end{align*}
    
    con lo que se demuestra la tesis de inducción, y la suma es distributiva con la multiplicación.
\end{enumerate}