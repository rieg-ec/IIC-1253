\begin{algorithm}
    \caption{Encontrar el MCM entre i y j $\in$ $N$, con j $\geq$ i}
    \begin{algorithmic}
        \STATE {w = j}
        \WHILE{mod($w$, i) $\neq$ 0} 
            \STATE {$w = w + j$}
        \ENDWHILE
        \RETURN {$w$}
    \end{algorithmic}
\end{algorithm}
    
        
Para demostrar que el algoritmo es correcto debemos demostrar que (1) este termina, y (2) si termina, es correcto. 

\begin{enumerate}
    \item Para demostrar que el algoritmo se termina, tomemos $x$ como la cantidad de veces que se ha ejecutado el loop. Basta con fijarse en que $x$ nuncá será mayor a $i$, puesto que si $w = j*i$, $mod(w, i) = 0$ y el algoritmo se terminaría. Hemos encontrado una cota superior para $x$, y como este es creciente entonces sabemos que en algún momento el algoritmo termina. Podemos escoger como invariante $w =  j*x$ con $x$ la cantidad de veces que se ejecute el loop. 
    
    Demostración:
    \begin{itemize}[label={}]
      \item \textbf{B.I.}: si $x = 1$, $w = j = 1 j$.
      \item \textbf{H.I.}: Supongamos que para $x_n$ cualquiera, $w = j * x_n$.
      \item \textbf{T.I}: luego de la iteración $x_n$, tenemos que $w_n = j * x_n$. La siguiente iteración, $w_{n+1} = w_n + j = j*x_n + j = x_{n+1} * j$. 
    \end{itemize}
    
    Y de esta forma nuestro invariante se mantiene independiente del número de iteraciones.
    
    \item Para demostrar que cuando acaba, este es correcto, debemos demostrar que nuestro invariante es el mínimo común multiplo entre $i$ y $j$. Por contradicción, supongamos que este NO es el MCM entre estos dos números. Esto significa que $\exists k$, t.q. $k < w$ y $k$ es el MCM entre $i$ y $j$. En ese caso, la condición del $while$ no se cumpliría, pues $mod(k, i) = 0$ puesto que $k$ es divisible por $i$ y el algoritmo se habría acabado retornando $w = k$. De esta forma, el invariante es siempre el menor número divisible por $i$ y $j$.
\end{enumerate}

Para calcular la complejidad del algoritmo, nos basaremos en $i$, es decir el menor número que recibe. Es facil ver que en el mejor caso, $mod (j, i) = 0$, por lo tanto $mod (j * 1, i ) = mod(w, i) = 0$ y el algoritmo solamente toma 3 pasos independiende del tamaño de los inputs. Así, en el mejor de los casos nuestro algoritmo es $O(1)$. En el peor de los casos, nos tomará $i$ repeticiones del loop llegar al MCM. De esa forma nuestro algoritmo es $O(i)$ en el peor de los casos. Si ocupamos el numero de digitos de $i$, una buena aproximación sería $d = log_{10} i$ con $d$ número de digitos. si escribimos $i$ en términos del tamaño de digitos, $i \approx 10^d$, por lo que nuestro algoritmo es $O(10^d)$ en el peor de los casos, es decir se comporta de manera exponencial con respecto al número de digitos de $i$.
