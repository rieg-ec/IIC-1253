%!TEX root =./tarea4.tex
\begin{enumerate}[a)]
    \item Si definimos $C \in S$ como el conjunto de máxima cardinalidad posible de números no consecutivos, este será de tamaño $n$. Está definido de tal manera que no podemos añadir otro elemento en $S \setminus C$ a $C$, pues dejaría de ser no consecutivo. Ahora intentamos armar el conjunto $X$ a partir de $C$, de manera que $X$ sea no consecutivo. Como $\vert X \vert = n + 1$, no podemos mapear cada elemento de $C$ a $X$, por el principio del palomar (pues no existe una función sobreyectiva desde $C$ a $X$), por lo tanto nos faltaría agregar un elemento al conjunto $X$ para que su cardinalidad sea $n+1$. Pero como ya no quedan elementos del conjunto $C$, debemos agregar elementos de $S \setminus C$, con lo que $X$ dejaría de ser no consecutivo, y tendría almenos un par de números consecutivos. 

   
\end{enumerate}
