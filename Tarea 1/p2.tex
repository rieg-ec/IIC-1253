%!TEX root =./tarea1.tex
Dado un conjunto $A$, definimos el conjunto de los $A$-remolinos $\mathcal{R}_A$
como el menor conjunto que cumple las siguientes reglas:

\begin{enumerate}
    \item $\forall x \in A, x \in \mathcal{R}_A$

    \item $\forall x, y \in A$

    \begin{itemize}
        \item $ x \lrinline y \in \mathcal{R}_A$, $ y \lrinline x \in \mathcal{R}_A$

        \item
           $\begin{matrix}
           x \\
           | & \in \mathcal{R}_A\\
           y \\
         \end{matrix}$,
         $\begin{matrix}
         y \\
         | & \in \mathcal{R}_A\\
         x \\
       \end{matrix}$
    \end{itemize}

    Para las siguientes reglas considere que $R \in \mathcal{R}_A$ y que $x,y \in A$.
    \item Sea un $A$-remolino de la forma $R \lrinline x$. Todos los siguientes son $A$-remolinos:

        $$R \lrinline x \lrinline y
        \hspace{15mm}
        \begin{matrix}
           R & \lrmatrix & x \\
             &   & |\\
             &   & y
         \end{matrix}
         \hspace{15mm}
         \begin{matrix}
             &   & y \\
             &   & |\\
           R & \lrmatrix & x
         \end{matrix}$$
    \item Sea un $A$-remolino de la forma $x \lrinline R$. Todos los siguientes son $A$-remolinos:
      $$y \lrinline x \lrinline R
        \hspace{15mm}
        \begin{matrix}
          x & \lrmatrix & R \\
          | &   &\\
          y &   &
        \end{matrix}
        \hspace{15mm}
        \begin{matrix}
          y  &   &\\
          |  &   &\\
          x & \lrmatrix & R
         \end{matrix}$$
    \item Sea un $A$-remolino de la forma $\begin{matrix}R \\ | \\ x \end{matrix}$.
    Todos los siguientes son $A$-remolinos:
      $$\begin{matrix}
        R\\
        |\\
        x\\
        |\\
        y
      \end{matrix}
        \hspace{15mm}
        \begin{matrix}
          R  &   &\\
          |  &   &\\
          x & \lrmatrix & y
         \end{matrix}
        \hspace{15mm}
        \begin{matrix}
            &   & R \\
            &   & |\\
          y & \lrmatrix & x
        \end{matrix}$$
    \item Sea un $A$-remolino de la forma $\begin{matrix}x \\ | \\ R \end{matrix}$.
    Todos los siguientes son $A$-remolinos:
      $$\begin{matrix}
        y\\
        |\\
        x\\
        |\\
        R
      \end{matrix}
        \hspace{15mm}
        \begin{matrix}
          x & \lrmatrix & y \\
          | &   &\\
          R &   &
        \end{matrix}
        \hspace{15mm}
        \begin{matrix}
           y & \lrmatrix & x \\
             &   & |\\
             &   & R
         \end{matrix}$$
\end{enumerate}
A modo de ejemplo, el siguiente es un $\N$-remolino:
$$ R \quad = \quad
  \begin{matrix}
    &           & 3  & \lrmatrix & 2 &           & 4 &           & 100\\
    &           & |  &           &   &           & | &           & |\\
  8 & \lrmatrix & 3  &           &   &           & 0 & \lrmatrix & 3\\
    &           & |  &           &   &           & | &           &\\
    &           & 27 & \lrmatrix & 1 & \lrmatrix & 1 &           &\\
    &           & |  &           & | &           &   &           &\\
    &           & 4  &           & 5 &           &   &           &
  \end{matrix}
$$
\begin{enumerate}
  \item[a)] Defina la función $size: \mathcal{R}_A \rightarrow \N$, la que
  recibe un $A$-remolino y retorna el número de elementos de $A$ que contiene.
  En el caso del ejemplo anterior, $size(R) = 13$.
  \item[b)] Considere la siguiente definición inductiva para el \textit{origen}
  de un $A$-remolino:

  $origin: \mathcal{R}_A \rightarrow A$
  \begin{enumerate}
    \item[1.] $\forall x \in A, origin(x) = x$
    \item[2.] $\forall x, y \in A$
    \begin{itemize}
        \item $ origin(x \lrinline y) = x$, $origin(y \lrinline x) = x$

        \item
           $origin\left(\begin{matrix}
           x\\
           |\\
           y
         \end{matrix}\right) = x$,
         $origin\left(\begin{matrix}
         y\\
         |\\
         x
       \end{matrix}\right) = x$
    \end{itemize}
    Para las siguientes reglas considere que $R \in \mathcal{R}_A$ y que $x \in A$.
    \item[3.] $origin(R \lrinline x) = origin(R)$.
    \item[4.] $origin(x \lrinline R) = origin(R)$.
    \item[5.] $origin\left(\begin{matrix}R \\ | \\ x \end{matrix}\right) = origin(R)$.
    \item[6.] $origin\left(\begin{matrix}x \\ | \\ R \end{matrix}\right) = origin(R)$.
  \end{enumerate}
  Note que con esta definición, un $A$-remolino puede tener múltiples orígenes.

  Demuestre que el número de orígenes de un $A$-remolino $R$ es igual a $size(R)$.
\end{enumerate}