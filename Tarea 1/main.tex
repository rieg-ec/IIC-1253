% Plantilla para documentos LaTeX para enunciados
% Por Pedro Pablo Aste Kompen - ppaste@uc.cl
% Licencia Creative Commons BY-NC-SA 3.0
% http://creativecommons.org/licenses/by-nc-sa/3.0/

\documentclass[12pt]{article}

% Margen de 1 pulgada por lado
\usepackage{fullpage}
% Incluye gráficas
\usepackage{graphicx}
% Packages para matemáticas, por la American Mathematical Society
\usepackage{amssymb}
\usepackage{amsmath}
% Desactivar hyphenation
\usepackage[none]{hyphenat}
% Saltar entre párrafos - sin sangrías
\usepackage{parskip}
% Español y UTF-8
\usepackage[spanish]{babel}
\usepackage[utf8]{inputenc}
% Links en el documento
\usepackage{hyperref}
\usepackage{fancyhdr}
\setlength{\headheight}{15.2pt}
\setlength{\headsep}{5pt}
\pagestyle{fancy}

\newcommand{\N}{\mathbb{N}}
\newcommand{\Exp}[1]{\mathcal{E}_{#1}}
\newcommand{\List}[1]{\mathcal{L}_{#1}}
\newcommand{\EN}{\Exp{\N}}
\newcommand{\LN}{\List{\N}}

\newcommand{\comment}[1]{}
\newcommand{\lb}{\\~\\}
\newcommand{\eop}{_{\square}}
\newcommand{\hsig}{\hat{\sigma}}
\newcommand{\ra}{\rightarrow}
\newcommand{\lra}{\leftrightarrow}

\newcommand\myeq{\stackrel{\mathclap{\smallfont\mbox{HI}}}{=}}

% Cambiar por nombre completo + número de alumno
\newcommand{\alumno}{Ramón Echeverría - 19638485}
\rhead{Tarea 1 - \alumno}

\begin{document}
\thispagestyle{empty}
% Membrete
% PUC-ING-DCC-IIC1103
\begin{minipage}{2.3cm}
\includegraphics[width=2cm]{logo.pdf}
\vspace{0.5cm} % Altura de la corona del logo, así el texto queda alineado verticalmente con el círculo del logo.
\end{minipage}
\begin{minipage}{\linewidth}
\textsc{\raggedright \footnotesize
Pontificia Universidad Católica de Chile \\
Departamento de Ciencia de la Computación \\
IIC1253 - Matemáticas Discretas \\}
\end{minipage}


% Titulo
\begin{center}
\vspace{0.5cm}
{\huge\bf Tarea 1}\\
\vspace{0.2cm}
\today\\
\vspace{0.2cm}
\footnotesize{2º semestre 2020 - Profesores G. Diéguez - F. Suárez}\\
\vspace{0.2cm}
\footnotesize{\alumno}
\rule{\textwidth}{0.05mm}
\end{center}



\section*{Respuestas}
% Estas numeracion es solo de ejemplo

\subsection*{Problema 1.a}


Demostraremos por inducción simple que $b_n = 3 \cdot 2^n \cdot n!$:

\vspace{5mm}

\textbf{BI}: nuestro caso base es $n = 0$:

\begin{equation*}
    \begin{aligned}
        3 \cdot 2^0 \cdot 0! = 3 = b_0
    \end{aligned}
\end{equation*}


\textbf{HI}: supongamos que $\forall n \in \N$ se cumple que $b_n = 3 \cdot 2^n \cdot n!$.

\textbf{TI}: demostraremos que lo anterior también se cumple para $n + 1$ $\in \N$:

\begin{equation*}
    \begin{aligned}
        b_{n+1} &= 2 \cdot (n + 1) \cdot b_n \\
                &= 2 \cdot (n + 1) \cdot 3 \cdot 2^n \cdot n! \\
                &= 3 \cdot 2^{n+1} \cdot (n + 1)!
    \end{aligned}
\end{equation*}

así, queda demostrado que se cumple para todo n $\in$ $\N$.


\newpage

\subsection*{Problema 1.b}

Demostraremos que todo $x$ $\in P$ tiene factorizacion p-prima ocupando inducción fuerte:

\textbf{BI}: el primer elemento del conjunto $P$ es el 2, el cual por ser p-primo tiene factorización p-prima.

\textbf{HI}: supongamos que $\forall$ $k < x$, $k$ $\in P$, se cumple que $k$ tiene factorización p-prima. Demostraremos que $x$ $\in P$ también tiene factorización p-prima: 

$x$ puede ser escrito como $2 \cdot a$ con $a \geq 2$

así, para a: 

(1) si  $a \notin P$, $x$ es p-primo por lo tanto posee factorización p-prima. 

(2) si $a \in P$, por nuestra HI sabemos que $a$ posee factorización p-prima. Así, x posee factorización p-prima también.

\newpage

\subsection*{Problema 2.a}

Para definir la función $size$ con inducción estructural, partiremos por el caso base:

\begin{itemize}
    \item Sea un conjunto A, para $x$ $\in A$, definimos $size(x) = 1$
    
    Sea el mismo conjunto A, con $x$ $\in A$ y sea $R_A$ remolino: 
    
    \item $size(R - x) = size(R) + 1$
    \item $size (x - R) = size(R) + 1$
    \item 
           $size\left(\begin{matrix}
           R\\
           |\\
           x
         \end{matrix}\right) = size(R) + 1$
    \item 
           $size\left(\begin{matrix}
           x\\
           |\\
           R
         \end{matrix}\right) = size(R) + 1$
\end{itemize}

\newpage

\subsection*{Problema 2.b}

Para la demostración primero definiremos la función originCount(R), la que nos retornará el numero de origenes de un A-remolino. 

Sea $A$ un conjunto cualquiera, $x \in A$ y $R_A$ remolino:
\begin{itemize}
    \item $originCount(x) = 1$
    \item $originCount(R - x) = originCount(R) + 1$
    \item 
           $originCount\left(\begin{matrix}
           R\\
           |\\
           x
         \end{matrix}\right) = originCount(R) + 1$
    \item 
           $originCount\left(\begin{matrix}
           x\\
           |\\
           R
         \end{matrix}\right) = originCount(R) + 1$
\end{itemize}

Ahora haremos la demostración ocupando inducción simple.

Sea $A$ un conjunto cualquiera, $x \in A$ y $R_A$ remolino:

\textbf{BI}: $size(x) = 1 = originCount(x)$.

\textbf{HI}: supondremos que se cumple para $R$, es decir, $size(R) = originCount(R)$

\textbf{TI}:
    \begin{itemize}
        \item $size(R - x) = size(R) + 1 \myeq originCount(R) + 1 = originCount(R - x)$
        \item $size(x - R) = size(R) + 1 \myeq originCount(R) + 1 = originCount(x - R)$
        \item 
           $size\left(\begin{matrix}
           x\\
           |\\
           R
         \end{matrix}\right) = size(R) + 1 \myeq originCount(R) + 1 = originCount\left(\begin{matrix}
           x\\
           |\\
           R
         \end{matrix}\right)$
         
         \item 
           $size\left(\begin{matrix}
           R\\
           |\\
           x
         \end{matrix}\right) = size(R) + 1 \myeq originCount(R) + 1 = originCount\left(\begin{matrix}
           R\\
           |\\
           x
         \end{matrix}\right)$
    \end{itemize}
    
    Queda demostrado que $size(R) = originCount(R)$, por lo tanto el número de origenes es igual a la función $size(R)$.

% Fin del documento
\end{document}
