% Plantilla para documentos LaTeX para enunciados
% Por Pedro Pablo Aste Kompen - ppaste@uc.cl
% Licencia Creative Commons BY-NC-SA 3.0
% http://creativecommons.org/licenses/by-nc-sa/3.0/

\documentclass[12pt]{article}

% Margen de 1 pulgada por lado
\usepackage{fullpage}
% Incluye gráficas
\usepackage{graphicx}
% Packages para matemáticas, por la American Mathematical Society
\usepackage{amssymb}
\usepackage{amsmath}
% Desactivar hyphenation
\usepackage[none]{hyphenat}
% Saltar entre párrafos - sin sangrías
\usepackage{parskip}
% Español y UTF-8
\usepackage[spanish]{babel}
\usepackage[utf8]{inputenc}
% Links en el documento
\usepackage[hidelinks]{hyperref}
\usepackage[shortlabels]{enumitem}

\usepackage{algorithmic}
\usepackage[nothing]{algorithm}

\newcommand{\bcap}{\ \hat{\cap} \ }
\newcommand{\bcup}{\ \hat{\cup} \ }
\newcommand{\bagP}{\hat{\mathcal{P}}}
\newcommand{\la}{\leftarrow}


\newcommand{\N}{\mathbb{N}}
\newcommand{\Z}{\mathbb{Z}}
\newcommand{\Exp}[1]{\mathcal{E}_{#1}}
\newcommand{\List}[1]{\mathcal{L}_{#1}}
\newcommand{\EN}{\Exp{\N}}
\newcommand{\LN}{\List{\N}}
\newcommand{\lrinline}{\text{\textbf{ --- }}}
\newcommand{\lrmatrix}{\text{\textbf{---}}}

\newcommand{\comment}[1]{}
\newcommand{\lb}{\\~\\}
\newcommand{\eop}{_{\square}}
\newcommand{\hsig}{\hat{\sigma}}
\newcommand{\ra}{\rightarrow}
\newcommand{\lra}{\leftrightarrow}
\newcommand{\com}{\textquotedbl}
\newcommand{\htau}{\hat{\tau}}
\newcommand{\twopartdef}[6]
{
	\left\{
		\begin{array}{ll}
			#1 &  #2 \\
			#3 &  #4 \\
			#5 &  #6
		\end{array}
	\right.
}

\newcommand{\mtwopartdef}[8]
{
	\left\{
		\begin{array}{ll}
			#1 &  #2 \\
			#3 &  #4 \\
			#5 &  #6 \\
			#7 &  #8
		\end{array}
	\right.
}

% Comentar aligns
\newcommand{\step}[1]{%
  \text{\phantom{(#1)}} \tag{#1}
}

% use 'pauta' to print pauta
\usepackage[pauta]{optional}

\begin{document}
\thispagestyle{empty}
% Membrete
% PUC-ING-DCC-IIC1103
\begin{minipage}{2.3cm}
\includegraphics[width=2cm]{logo.pdf}
\vspace{0.5cm} % Altura de la corona del logo, así el texto queda alineado verticalmente con el círculo del logo.
\end{minipage}
\begin{minipage}{\linewidth}
\textsc{\raggedright \footnotesize
Pontificia Universidad Católica de Chile \\
Departamento de Ciencia de la Computación \\
IIC1253 - Matemáticas Discretas \\}
\end{minipage}


% Titulo
\begin{center}
\vspace{0.5cm}
{\huge\bf Tarea 2}\\
\vspace{0.2cm}
\today \\
\vspace{0.2cm}
\footnotesize{2º semestre 2020 - Profesores G. Diéguez - F. Suárez}
\vspace{0.2cm}
\rule{\textwidth}{0.05mm}
\end{center}

% Cuerpo del documento ejemplo

\section*{Requisitos}
\begin{itemize}

  \item La tarea es individual. Los casos de copia serán sancionados con la
   reprobación del curso con nota $1.1$.

  \item \textbf{Entrega}: Hasta las 23:59:59 del 14 de septiembre a través del buzón habilitado en el sitio del curso (Canvas).
  Se habilitara un buzon distinto para cada pregunta para facilitar la correcion.

  \begin{itemize}

    \item Esta tarea debe ser hecha completamente en \LaTeX. Tareas hechas a mano o
    en otro procesador de texto \textbf{no serán corregidas}.

    \item Debe usar el template \LaTeX\ publicado en la página del curso.

    \item Cada problema debe entregarse en un archivo independiente de las demas preguntas.
    

    \item Los archivos que debe entregar son un archivo \texttt{PDF} por cada pregunta
    a su solución con nombre \texttt{numalumno-P1.pdf} y \texttt{numalumno-P2.pdf}, 
    junto con un \texttt{zip} con nombre \texttt{numalumno-P1.zip} y \texttt{numalumno-P2.zip}, 
    conteniendo el archivo \texttt{numalumno-P1.tex} y \texttt{numalumno-P2.tex},respectivamente, 
    que compila su tarea. 
    Si su código hace referencia a otros archivos, debe incluirlos también.


  \end{itemize}

  \item El no cumplimiento de alguna de las reglas se penalizará con un descuento de 0.5 en la nota final (acumulables).

  \item No se aceptarán tareas atrasadas.

  \item Si tiene alguna duda, el foro de Canvas es el lugar oficial para realizarla.
\end{itemize}



\section*{Problemas}

    \subsection*{Problema 1 - Lógica proposicional}
      %!TEX root =./tarea2.tex

\newcommand\myeq{\stackrel{\mathclap{\scriptsize\mbox{HI}}}{\equiv}}

Dado $\alpha, \beta \in L(P)$, considere el siguiente conectivo binario:

\[ \sigma(\alpha \sim \beta) = 
\begin{cases}
    1 & \text{ si $\sigma(\alpha) = 0$ y $\sigma(\beta)= 0$} \\
    0 & \text{ en otro caso}
\end{cases} 
\]

\begin{enumerate}[a)]
  \item ¿Es $\sim$ conmutativo? Demuestre.
  \item ¿Es $\sim$ asociativo? Demuestre.
  \item ¿Es $\sim$ funcionalmente completo? Demuestre.
\end{enumerate}

\subsubsection*{Respuestas:}

\begin{enumerate}[a)]
    \item Para probar que $\sim$ es conmutativo compararemos las tablas de verdad de $\alpha \sim \beta$ y $\beta \sim \alpha$:
    
    \begin{center}
        \begin{tabular}{c|c|c|c|c}
            & $\alpha$ & $\beta$ & $\alpha \sim \beta$ & $\beta \sim \alpha\\
            \hline
            $\sigma_1$: & $0$ & $0$ & $1$ & $1$ \\
            $\sigma_2$: & $0$ & $1$ & $0$ & $0$ \\
            $\sigma_3$: & $1$ & $0$ & $0$ & $0$ \\
            $\sigma_4$: & $1$ & $1$ & $0$ & $0$ \\
        \end{tabular}
    \end{center}
    
    Como $\alpha \sim \beta \equiv \beta \sim \alpha$ el conectivo es conmutativo.
    
    \item Para probar que $\sim$ no es asociativo, basta encontrar un caso: tomemos $\alpha = 0$, $\beta = 0$ y $\gamma = 1$:
    
    \hspace{10mm} $\alpha \sim (\beta \sim \gamma) = 1$
    
    \hspace{10mm} $(\alpha \sim \beta) \sim \gamma = 0$
    
    Por lo que el conectivo no es asociativo.
    
    \item Se puede demostrar que el conectivo es funcionalmente completo si se encuentra una forma logicamente equivalente de escribir un conjunto funcionalmente completo ocupando solo el conectivo $\sim$. Sea el conjunto funcionalmente completo $C = \{\neg, \vee\}$, demostraremos que todo conectivo en $C$ puede ser escrito usando conectivos en $C' = \{\sim\}$ de forma inductiva:
    
    \textbf{BI}: Si $\varphi = p$ con $p \in P$, se cumple trivialmente.
    
    \textbf{HI}: Supongamos que $\varphi \equiv \varphi'$, $\psi \equiv \psi'$ con $\varphi, \psi$  $\in L(P)$ que ocupan solo conectivos en $C$, y $\varphi'$, $\psi'$ ocupan solo conectivos $C'$
    
    \textbf{TI}: Aplicaremos el paso inductivo para definir $\theta$ formula que ocupa solo conectivos en $C$ y escribiremos una formula logicamente equivalente con conectivos en $C'$:
    
    \begin{enumerate}[label=\roman*.]
        \item $\theta = \neg \varphi \myeq \neg \varphi' \equiv \varphi' \sim \varphi'$
        \item $\theta = \varphi \vee \psi \myeq \varphi' \vee \psi' \equiv (\varphi' \sim \psi') \sim (\varphi' \sim \psi')$
    \end{enumerate}
    
    Por lo que queda demostrado que todo conectivo en $C'$ puede ser escrito ocupando solo conectivos en $C'$, y como $C$ es funcionalmente completo, $C'$ también lo es.
\end{enumerate}

    \newpage
     \subsection*{Problema 2 - Lógica de predicados}
      %!TEX root =./tarea4.tex
\begin{enumerate}[a)]
    \item Si definimos $C \in S$ como el conjunto de máxima cardinalidad posible de números no consecutivos, este será de tamaño $n$. Está definido de tal manera que no podemos añadir otro elemento en $S \setminus C$ a $C$, pues dejaría de ser no consecutivo. Ahora intentamos armar el conjunto $X$ a partir de $C$, de manera que $X$ sea no consecutivo. Como $\vert X \vert = n + 1$, no podemos mapear cada elemento de $C$ a $X$, por el principio del palomar (pues no existe una función sobreyectiva desde $C$ a $X$), por lo tanto nos faltaría agregar un elemento al conjunto $X$ para que su cardinalidad sea $n+1$. Pero como ya no quedan elementos del conjunto $C$, debemos agregar elementos de $S \setminus C$, con lo que $X$ dejaría de ser no consecutivo, y tendría almenos un par de números consecutivos. 

   
\end{enumerate}


\end{document}