%!TEX root =./tarea2.tex

\newcommand\myeq{\stackrel{\mathclap{\scriptsize\mbox{HI}}}{\equiv}}

Dado $\alpha, \beta \in L(P)$, considere el siguiente conectivo binario:

\[ \sigma(\alpha \sim \beta) = 
\begin{cases}
    1 & \text{ si $\sigma(\alpha) = 0$ y $\sigma(\beta)= 0$} \\
    0 & \text{ en otro caso}
\end{cases} 
\]

\begin{enumerate}[a)]
  \item ¿Es $\sim$ conmutativo? Demuestre.
  \item ¿Es $\sim$ asociativo? Demuestre.
  \item ¿Es $\sim$ funcionalmente completo? Demuestre.
\end{enumerate}

\subsubsection*{Respuestas:}

\begin{enumerate}[a)]
    \item Para probar que $\sim$ es conmutativo compararemos las tablas de verdad de $\alpha \sim \beta$ y $\beta \sim \alpha$:
    
    \begin{center}
        \begin{tabular}{c|c|c|c|c}
            & $\alpha$ & $\beta$ & $\alpha \sim \beta$ & $\beta \sim \alpha\\
            \hline
            $\sigma_1$: & $0$ & $0$ & $1$ & $1$ \\
            $\sigma_2$: & $0$ & $1$ & $0$ & $0$ \\
            $\sigma_3$: & $1$ & $0$ & $0$ & $0$ \\
            $\sigma_4$: & $1$ & $1$ & $0$ & $0$ \\
        \end{tabular}
    \end{center}
    
    Como $\alpha \sim \beta \equiv \beta \sim \alpha$ el conectivo es conmutativo.
    
    \item Para probar que $\sim$ no es asociativo, basta encontrar un caso: tomemos $\alpha = 0$, $\beta = 0$ y $\gamma = 1$:
    
    \hspace{10mm} $\alpha \sim (\beta \sim \gamma) = 1$
    
    \hspace{10mm} $(\alpha \sim \beta) \sim \gamma = 0$
    
    Por lo que el conectivo no es asociativo.
    
    \item Se puede demostrar que el conectivo es funcionalmente completo si se encuentra una forma logicamente equivalente de escribir un conjunto funcionalmente completo ocupando solo el conectivo $\sim$. Sea el conjunto funcionalmente completo $C = \{\neg, \vee\}$, demostraremos que todo conectivo en $C$ puede ser escrito usando conectivos en $C' = \{\sim\}$ de forma inductiva:
    
    \textbf{BI}: Si $\varphi = p$ con $p \in P$, se cumple trivialmente.
    
    \textbf{HI}: Supongamos que $\varphi \equiv \varphi'$, $\psi \equiv \psi'$ con $\varphi, \psi$  $\in L(P)$ que ocupan solo conectivos en $C$, y $\varphi'$, $\psi'$ ocupan solo conectivos $C'$
    
    \textbf{TI}: Aplicaremos el paso inductivo para definir $\theta$ formula que ocupa solo conectivos en $C$ y escribiremos una formula logicamente equivalente con conectivos en $C'$:
    
    \begin{enumerate}[label=\roman*.]
        \item $\theta = \neg \varphi \myeq \neg \varphi' \equiv \varphi' \sim \varphi'$
        \item $\theta = \varphi \vee \psi \myeq \varphi' \vee \psi' \equiv (\varphi' \sim \psi') \sim (\varphi' \sim \psi')$
    \end{enumerate}
    
    Por lo que queda demostrado que todo conectivo en $C'$ puede ser escrito ocupando solo conectivos en $C'$, y como $C$ es funcionalmente completo, $C'$ también lo es.
\end{enumerate}
